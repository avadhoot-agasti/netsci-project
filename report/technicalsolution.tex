\section{Technical Solution} \label{techsoln}

\subsection{Data Sources} \label{datasources}
 Our approach is to identify the dependent packages by looking
 at the dependencies using \textit{pip show} output. For example
 \begin{lstlisting}
pip show networkx

Name: networkx

Version: 1.11

Summary: Python package for creating and 
manipulating graphs and networks

Home-page: http://networkx.github.io/

Author: NetworkX Developers

Author-email: networkx-discuss@googlegroups.com

License: BSD

Location: /Libs/anaconda/lib/python3.6/site-packages

Requires: decorator
 \end{lstlisting}
 Looking at the above output we can clearly see that \textit{networkxx}
 depends on package \textit{decorator} and further package \textit{decorator}
 may be dependent on other package and so on. We continue to traverse
 we should be able to find all dependent packages till we reach code python
 libraries. Interestingly, there \textit{128k} python packages available
 \cite{www-python-org} which should give us a lot of data points for our
 analysis.

\subsection{Automation} \label{automation}
 We can automate running this analysis on these packages using python script. After 
 collecting this information each package will form a node in the graph and each
 edge will represent the dependency with the next module. We should be able
 to plot this graph using \textit{networkx} module and represent the most important nodes
 which become the hubs to the network. Also, compute the degree and clustering 
 co-efficient for this graph. Further, we can also study to graph to identify if the graph 
 follows the \textit{power law} distribution or it is a \textit{scale free} network or its
 just a \textit{random} network.

\subsection{Technology} \label{tech}
 We plan to perform most of our coding in Python itself. We plan to use Gephi
  for visualizing the network.


\subsection{Challanges} \label{techchallanges}
During this exercise we can run into few technical challenges for example
to perform this analysis we have to install each python package locally
which could be really resource intensive specially in terms of disk space.
Also, while doing the analysis and loading the graph for 128k packages
could be memory and cpu intensive. To get around this problem we may 
have to reduce the analysis to less number of python packages and keep
adding mode packages as we make progress.