\section{Related Work} \label{relwork}
\begin {itemize}
\item
The project Pipedtree - \cite{www-pipdeptree} is command
line utility which allows user to see the installed packages in the form of
dependency tree. However, this utility is focused more on solving dependency
conflicts than the kind of analysis we propose to perform.

\item
Analysis of 30K Github projects \cite{www-takipi}.
The project was aimed to analyze the 30k different Java, Ruby and Javascript
projects on Github to understand the top libraries being used. While their
analysis was similar to what we plan to do, the approach was not network based.

\item
This paper \textit{Power Laws in Software} \cite{louridas2008power} 
does analysis of power law distribution in a software application at class 
level and function level. It did analysis of java, perl, c/c++ etc applications 
and did establish a pattern that  these applications do follow power law distribution

\end {itemize}

\section{Further Enhancement} \label{enhancements}
Further enhancements can be done to crawler stage where instead of relying on
modules installed locally it can crawl on module available on pypi.org. This will 
reduce an additional step of installing the modules locally and save the disk space
as well as more flexible to crawl on web. Another enhancement can be made to make it
more generic to do the analysis on any given language for example java maven 
dependencies or perl packages or ruby version manager and packages etc
