\section{Introduction} \label{intro}
When we try to install a python library, or rather when we use a python
library first thing that we need to do is to \textit{pip install} it.
Most of the times, a python library is dependent on another library. Our idea
 is to build a network or graph of the python library dependency. In this
 network, every library will be a node of the network and the dependency
 will form an edge.
 Such a graph can be queried to answer several questions like
\begin{itemize}
\item Which are the most core packages which are widely, directly or
undirectly, used in large number of other packages?
\item In which subject-area the new development is happening. We can pivot
this solution around the small number of packages which are included in large
 number of packages. For example, if networkX is being used by lot of new
 packages then we can say that there is lot of development happening in
 network science
\item What all packages will be impacted due to changes in a base package (e
.g. if we find a severe bug in networkX, what are the other packages which
can be potentially impacted due to the bug fix)
\end{itemize}
  