\documentclass[sigconf]{acmart}
%\settopmatter{printacmref=false} % Removes citation information below abstract
\renewcommand\footnotetextcopyrightpermission[1]{} % removes footnote with conference information in first column
\pagestyle{plain} % removes running headers

\usepackage{booktabs} % For formal tables
\usepackage{listings}


\begin{document}
\title{Analyzing small world phenomenon and scale free network of python packages}

\author{Avadhoot Agasti}
\email{aagasti@indiana.edu}

\author{Abhishek Gupta}
\email{abhigupt@iu.edu}

\begin{abstract}

The \textit{small world} phenomenon is a very important and interesting concept which can be practically 
used in optimizing imports for programming languages like python, java etc. This will also help us understand
the network of the packages used by developers and classify them as scale free or random network. Further,
this analysis can be extended to any other programming language and understand the package network. The 
scope of this project is to focus on python based packages and navigate to all dependent packages to build
a network graph of these packages. Further, identify the important packages which could be like important
nodes in scale-free network.   
\end{abstract}


\keywords{scale-free, random networks, small-world, packages}

\maketitle

\section{Introduction} \label{intro}
When we try to install a python library, or rather when we use a python
library first thing that we need to do is to \textit{pip install} it.
Most of the times, a python library is dependent on another library. Our idea
 is to build a network or graph of the python library dependency. In this
 network, every library will be a node of the network and the dependency
 will form an edge.
 Such a graph can be queried to answer several questions like
\begin{itemize}
\item Which are the most core packages which are widely, directly or
undirectly, used in large number of other packages?
\item In which subject-area the new development is happening. We can pivot
this solution around the small number of packages which are included in large
 number of packages. For example, if networkX is being used by lot of new
 packages then we can say that there is lot of development happening in
 network science
\item What all packages will be impacted due to changes in a base package (e
.g. if we find a severe bug in networkX, what are the other packages which
can be potentially impacted due to the bug fix)
\end{itemize}
  
\section{Requirements} \label{req}
The goal of this analysis is
 \begin{itemize}
 \item to create a network graph of these dependencies
 \item understand the dependencies and important nodes
 \item identify if its a scale-free network
 \item identify any other properties depicted by this graph
 \item update the graph and build the graph as and when needed
 \end{itemize}
 At the end of this exercise we should be able to run this
 analysis on available python \cite{www-python-org} packages.
 Also, we need to understand and compute the other properties of the graph
 like average degree, average clustering co-efficient, average path length etc
 Also verify if the graph is \textit{scale free} or just a \textit{random} graph.  

\section{Technical Solution} \label{techsoln}


\input{visualizations}
\input{viz_top_countries}
\input{viz_top_agents}
\input{viz_top_refs}
\input{viz_top_searches}
\input{viz_top_urls}
\input{viz_events_traffic}

\input{similartechnologies}

\input{conclusion}

%\end{document}  % This is where a 'short' article might terminate


\section{Acknowledgements}
 The authors thank Prof. Katy Borner for her technical guidance. The
 authors would also like to thank TAs of Information Visualization class for their valued
 support.

\bibliographystyle{plain}
\bibliography{references}

\end{document}
